\chapter{Info importante que NO va en el texto}

\begin{itemize}
    \item \hl{15 mg of each sample was heated from room temperature
to 560°C under nitrogen atmosphere at
heating rate of 10°C/min to monitor the weight loss
of oil and elastomers. At 560°C, the gas flow was
changed to oxygen atmosphere and the samples
were heated from 560 to 800°C in order to observe
the carbon black degradation. The experiments were performed
according to the following program: (A) Initial temperature
of 30°C; (B) Cooling: cooling rate of
30°C min–1 to –90°C; (C) Heating: heating rate of
30°C min–1 to 250°C. The results reported in this
work correspond to the heating runs. All DSC curves
were normalized according to the sample mass.}

\item \hl{The soluble (sol) and insoluble (gel) fractions of
each sample, after the devulcanization process, were
determined by Soxhlet extraction, by using toluene
about 4 g of GTR was immersed in toluene for
6 h at 80°C. After extraction, the samples were
dried for 24 h at 80°C to remove the solvent and
their masses were measured}

\item In another study,
chemical and mechano-chemical methods were used by Rooj et al. [51] for devulcanizing natural
rubber (NR) utilizing benzoyl peroxide as a devulcanization aid. As the outcomes revealed, sulfur
crosslink happened selectively at the optimum temperature and optimum time of 80 °C and 6 h,
respectively. 

\item sol-gel measurements
acetone was used as the extraction solvent for a minimum time of 12 h before performing
the swelling test. Then, the samples were dehumidified in an oven until reaching a constant weight A small amount of sample was immersed in toluene for 3 days at room temperature.
Afterward, it was placed in an oven at 80 \unit{\degreeCelsius} C until reaching a constant weight


\item Crosslink density measurements
Crosslink density of the gel part of the virgin sample and samples devulcanized by DSO and
TMTD was measured by the swelling test utilizing toluene solvent. The crosslinking density was
determined using the Fluorine-Renner equation (Equation 2), and the interaction parameter ()
assumed to be 0.45 for the toluene-rubber system

\item At 80 C, benzoyl peroxide undergoes free
radical chain scission, which produces highly unstable
benzoyl radicals. Due to its instability, it spontaneously
reacts with the sulfur present in the cured rubber. It is well
known that the bond energy of S–S (227 kJ/mol) and C–S
bonds (273 kcal/mol) is lower than that of the C–C bond
(348 kJ/mol)

The bond energy of different chemical bonds is as
follows: 348 kJ/mol for C–C, 621 kJ/mol for C=C, 416 kJ/
mol for C–H, 273 kJ/mol for C–S, 227 kJ/mol for S–S and
378 kJ/mol for C–C= in CH3-CH=CH2 compound

\end{itemize}

\subsection{Posibles normas a leer}

Reologia: ASTM D-2084-11
SWELLING TEST: ASTM D6814-02
Mooney viscosimeter ASTM D1646.
Free sulfur ASTM D297-72A
Mechanical properties ASTM D 412
Durometer ASTM D 2240 


\chapter{Resultados}

\begin{multicols}{2}

\subsection{Caracterización del GTR}

Para caracterizar el material GTR original, se realizaron los siguientes análisis:  
\begin{itemize}
    \item TGA  
    \item DSC  
    \item FTIR  
    \item Reometría  
\end{itemize}

\subsection{Desvulcanización}

El efecto de la radiación por microondas sobre el material se evaluó sometiéndolo a diferentes tiempos de irradiación: x, x, x, x y x \hl{segundos}, y variando el porcentaje de agente desvulcanizante. Durante el proceso, se registraron los cambios físicos observados, como el color del humo (en caso de generarse), la fluidez del material, la descomposición, y se midió la temperatura dentro del equipo. % Cabe destacar que algunas muestras podrían destruirse o incendiarse durante el experimento.

\subsection{Caracterización del GTR con agentes desvulcanizantes}

\subsubsection{Antes de la radiación por microondas}

Las mezclas de GTR con agentes desvulcanizantes se caracterizaron mediante:  
\begin{itemize}
    \item TGA  
    \item DSC  
    \item FTIR  
    \item Reometría  
\end{itemize}

\subsubsection{Después de la radiación por microondas}

Posterior a la radiación, las muestras se caracterizaron utilizando las mismas técnicas:  
\begin{itemize}
    \item TGA  
    \item DSC  
    \item FTIR  
    \item Reometría  
\end{itemize}

\end{multicols}

\subsection{Resultados Obtenidos}

\subsection{Resultados esperados \hl{acorde a la literatura}}

\begin{enumerate}
    \item parcial devulcanization for disulfide (-S2-) and poly-sulfide (-SX-) pero dificultad en romper enlaces C-S (monosulfidicos) debido a la energia proveida por las ondas microondas.
  
    %a temperatura de reacción, 

    \item Ojo las propiedades del cauco se disminuyen debido a la escisión mediante microondas!.
    Al aumentar el tiempo de reacción y la cantidad de agentes de desvulcanización, se observaría una disminución en la densidad de enlaces cruzados y un aumento en el porcentaje de desvulcanización. \hl{the higher the sol fraction of the samples
devulcanized by devulcanizing agent, the higher the devulcanization percentage}



\item by enhancing the \hl{enhancing devulcanazing agent content up to 5 phr, the volume fraction and crosslink density of rubber decrease while the sol fraction and devulcanization porcentaje increase}

\item FTIR
Disminucion en pico S-S (556), que no haya cambios en picos C-C (Stretching vibrations), lo que indica que (main polymer bonds have not been destroyed) como no hay grandes cambios en el espectro --> demonstrating that the breakdown of the main chains does not happen in
the polymer \cite{Zhang2024AnDevulcanization}.

\item temperature of 150 C and a pressure of 5 MPa for 8 min
in an electrically heated press

\end{enumerate}